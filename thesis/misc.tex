\section{Vandalism: Original Research}
According to Wikipedia's newspaper, the Signpost, edit filters were initially introduced as a vandalism prevention mechanism~\cite{Signpost2009}.
The aim of this section is to provide a better understanding of vandalism on Wikipedia: What is vandalism, and what not; who engages in vandalism; who is striving to prevent it and with what means?

\subsection{What is vandalism}

According to EN Wikipedia's policy~\cite{Wikipedia:Vandalism}, vandalism means ``intentionally making abusive edits to Wikipedia'' or, more specifically ``editing (or other behavior) deliberately intended to obstruct or defeat the project's purpose, which is to create a free encyclopedia''.
Vandalism includes ``malicious removal of encyclopedic content, or the changing of such content beyond all recognition, without any regard to our core content policies of neutral point of view (which does not mean no point of view), verifiability and no original research''
as well as ``adding irrelevant obscenities or crude humor to a page, illegitimately blanking pages, and inserting obvious nonsense into a page''
and ``[a]busive creation or usage of user accounts and IP addresses''.

Wikipedians have elaborated following vandalism typology~\cite{Wikipedia:Vandalism}:
\begin{itemize}
    \item  Abuse of tags
    \item  Account creation, malicious
    \item  Avoidant vandalism
    \item  Blanking, illegitimate
    \item  Copyrighted material, repeated uploading of
    \item  Edit summary vandalism
    \item  Format vandalism
    \item  Gaming the system
    \item  Hidden vandalism
    \item  Hoaxing vandalism
    \item  Image vandalism
    \item  Link vandalism
    \item  Page creation, illegitimate
    \item  Page lengthening
    \item  Page-move vandalism
    \item  Silly vandalism
    \item  Sneaky vandalism
    \item  Spam external linking
    \item  Stockbroking vandalism
    \item  talk page vandalism
    \item  Template vandalism
    \item  User and user talk page vandalism
    \item  Vandalbots
\end{itemize}

\subsection{What is not vandalism}

Additionally, there are different types of edits viewed as disruptive by the Wikipedia community.
Edit warring and pushing a single point of view and disregarding community feedback are examples here of. %TODO what are other examples?
Nevertheless, the guidelines warn that ``[d]isruptive editing is not vandalism, though vandalism is disruptive''~\cite{Wikipedia:DisruptiveEditing}.
And that different procedures should be adopted by editors in both cases.

The vandalism policy also cautions against using the ``vandalism'' label unless absolutely necessary since it tends to drive contributors away and prevent constructive discussions~\cite{Wikipedia:Vandalism}.
%TODO vgl good faith memo

\begin{comment}
\url{https://en.wikipedia.org/wiki/Wikipedia:Vandalism}
"Careful consideration may be required to differentiate between edits that are beneficial, edits that are detrimental but well-intentioned, and edits that are vandalism."
%TODO vgl with memo-good-faith

\url{https://en.wikipedia.org/wiki/Wikipedia:Disruptive_editing}

"Disruptive editing is not always intentional. Editors may be accidentally disruptive because they don't understand how to correctly edit, or because they lack the social skills or competence necessary to work collaboratively "
Okay what are disruptive edits that are not vandalism? (apart from edit wars)

"Engages in "disruptive cite-tagging"; adds unjustified {{citation needed}} tags to an article when the content tagged is already sourced, uses such tags to suggest that properly sourced article content is questionable."
\end{comment}

\subsection{Who engages in vandalism (and why?)}

The policy signals clearly that editors repeatedly engaging in vandalism are subject to banning.
Furthermore, it is explained that although warnings for vandalism are issued in general, these are not a prerequisite for banning~\cite{Wikipedia:Vandalism}.
%TODO: still not explained who and why

\subsection{Who is striving to prevent vandalism? How do they go about it?}

Since Wikipedia is a ``do-it-yourself'' project, every editor who notices vandalism is called upon to help fixing it.
There is a formal process for reporting users who persistently continue to engage in vandalism despite warnings~\cite{Wikipedia:AIV}, %TODO go into more detail?
as well as for requesting page protection for frequently vandalised pages~\cite{Wikipedia:PageProtection}.
And there are also users who specifically dedicate substantial amount of their Wikipedia contributions to fighting vandalism.

These dedicated vandal fighters mostly do so with the aid of some (semi or fully) automated tools which not only significantly speeds up the process (see below),
but, according to research, fundamentally changes the nature of the encyclopedia and its collaboration ecosystem~\cite{GeiRib2010}.

