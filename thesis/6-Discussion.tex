\chapter{Discussion and Limitations}
\label{chap:discussion}

\section{Discussion}

Discuss results:
so I've now explored and gathered understanding on Background(Context), general workings of the edit filter system and the state of the art of edit filters on the EN Wikipedia.
So what? What important/interesting insights have I gathered when contemplating all of this together?

* also comment on negative results!

* why get certain filters (and not others?)
* do filters solve effectively the task they were conjured up to life to fulfil?
* what kinds of biases/problems are there?
* who is allowed to edit edit filters?

Alternative approaches to community management:
compare with Surviving the Eternal September paper~\cite{KieMonHill2016}
"importance of strong
systems of norm enforcement made possible by leadership,
community engagement, and technology."

"emphasizing decentralized moderation" //all community members help enforce the norms
"ensuring enough leadership capacity is available
when an influx of newcomers is anticipated."
"Designers may
benefit by focusing on tools to let existing leaders bring others
on board and help them clearly communicate norms."
"designers should support an ecosystem of accessible and ap-
propriate moderator tools."

%***************************************

* a realisation: number of filters cannot grow endlessly, every edit is checked against all of them and this consumes computing power! (signaled in various places) (and apparently haven't been chucked with Moore's law). is this the reason why number of filters has been more or less constanst over the years?
* there seems to be a hard condition limit for filters: so the active ones are best of! which filters are best-of? a theory: "I've combated so and so many occurances of vandalism X with my bot. Let us implement a filter for this"

* Claudia thinks it's easier to implement a filter than a bot (less technical knowledge needed)
* Filter trigger before a publication, Bots trigger afterwads
  ** that's positive! editors get immmediate feedback and can adjust their (good faith) edit and publish it! which is psychologically better than publish something and have it reverted in 2 days
* thought: filter are human centered! (if a bot edits via the API, can it trigger a filter? Actually, I think yes, there were a couple of filters with something like "vandalbot" in their public comment)


\section{Limitations}

This work presents a first attempt at analysing Wikipedia's edit filter system.
It has several limitations (we could think of).
First, it focuses on English Wikipedia only.
We are convinced that there are valuable lessons to be learnt (about the communities, usefulness of filters, ..) from comparing edit filter use across different language versions.
Second, unfortunately, including an ethnographic analysis was not possible.
This is partially due to the fact that we employ a computer science perspective on the question and partially due to limited time.
Third, the manual filter classification was undertaken by one person only, so biases of this person have certainly shaped the labels.

%TODO describe also negative results!
