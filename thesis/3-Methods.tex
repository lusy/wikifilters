\chapter{Methods}
\label{chap:methods}

\section{Grounded Theory}
\section{Trace Ethnography}

\cite{GeiRib2011}
Introduce the methodology (and the concept) of trace ethnography.

Def
"combines the richness of participant-observation
with the wealth of data in logs so as to reconstruct
patterns and practices of users in distributed
sociotechnical systems."

"exploits the proliferation of
documents and documentary traces"

"traces not only
document events but are also used by participants
themselves to coordinate and render accountable many
activities"

"heterogeneous data – which include transaction logs,
version histories, institutional records, conversation
transcripts, and source code"
"allowing us to retroactively reconstruct specific actions
at a fine level of granularity"

"turn thin documentary traces into
“thick descriptions” [10] of actors and events"

"traces can only
be fully inverted through an ethnographic
understanding of the activities, people, systems, and
technologies which contribute to their production."

traditional ethnographic observation is costly and inpractical in distributed settings (and may miss phenomena that occur between sites)

Critique:
"it only can observe what the system
or platform records, which are always incomplete."

Concerns:
- ethical: breaching privacy via thickening the traces; no possibility for informed consent

\cite{GeiHal2017}
"when working with large-scale “found data” [36] of the traces
users leave behind when interacting on a platform, how do we best operationalize culturally-specific
concepts like conflict in a way that aligns with the particular context in which those traces were made?"

Star: "ethnography of infrastructure":
"discusses the “veridical” approach, in which “the information system
is taken unproblematically as a mirror of actions in the world, and often tacitly, as a complete
enough record of those actions” (p. 388).
She contrasts this with seeing the data as “a trace or record
of activities,” in which the information infrastructure “sits (often uneasily) somewhere between
research assistant to the investigator and found cultural artifact."

"Trace
ethnography is not “lurker ethnography” done by someone who never interviews or participates in
a community."
trace literacy --> get to know the community; know how to participate in it

thick description of different prototypical cases:

\begin{comment}
vgl \cite{GeiHal2017}
iterative mixed method
combination of:
* quantitative methods: mining big data sets/computational social science
"begin with one or
more large (but often thin) datasets generated by a software platform, which has recorded digital
traces that users leave in interacting on that platform. Such researchers then seek to mine as much
signal and significance from these found datasets as they can at scale in order to answer a research
question"
* more traditional social science/qualitative methods, e.g. interviews, observations, experiments
\end{comment}

\section{Cooking Data With Care}
or Critical data science? Or both?
