% ---------------------------------------------------
% ----- Abstract (English) of the template
% ----- for Bachelor-, Master thesis and class papers
% ---------------------------------------------------
%  Created by C. Müller-Birn on 2012-08-17, CC-BY-SA 3.0.
%  Freie Universität Berlin, Institute of Computer Science, Human Centered Computing.
%
\pagestyle{empty}

\subsection*{Abstract}

The present thesis offers a first investigation of one of Wikipedia's quality control mechanisms–edit filters.
It is analysed how edit filters fit in the quality control frame on English Wikipedia, why they were introduced and what tasks they take over.
Moverover, it is discussed why rule based systems seem to be still popular today, when more advanced machine learning methods are available.
It was found that edit filters seem to be responsible for obvious but persistent types of vandalism, gladly(am liebsten syn) disallowing these from the start so that (human) resources can be used more efficiently elsewhere.
In addition to disallowing this vandalism, edit filters seem (syn) to be applied in ambiguous situations where an edit is disruptive but motivation of the editor is not clear.
In these ambiguous (syn) cases, the filters take an ``assume good faith'' approach and seek via warning messages to guide the disrupting editor towards transforming their contribution to a constructive one.
There are also a smaller number of filters taking care of random (syn!) maintenance tasks–above all tracking certain bug or other behaviour for further investigation.
% Evaluation??

\begin{comment}
Include
- the topic of this thesis,
- important contents,
- results of your research
- and an evaluation of your results.
\end{comment}

\cleardoublepage
