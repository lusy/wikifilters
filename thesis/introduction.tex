% ---------------------------------------------------
% ----- Introduction of the template
% ----- for Bachelor-, Master thesis and class papers
% ---------------------------------------------------
%  Created by C. Müller-Birn on 2012-08-17, CC-BY-SA 3.0.
%  Last upadte: C. Müller-Birn 2015-11-27
%  Freie Universität Berlin, Institute of Computer Science, Human Centered Computing.
%
\chapter{Introduction}
\label{chap:introduction}

There's an inscription in Code 2.0:
"Code 2.0
TO WIKIPEDIA, THE ONE SURPRISE THAT TEACHES MORE THAN EVERYTHING HERE." (p.v)~\cite{Lessig2006}

Comment: Wikipedia is important because it teaches us stuff. We can look at it and see things and maybe infer how other, less open systems work.

\section{Genesis}

% When and why were Wikipedia edit filters introduced?

Edit filters were first introduced on the English Wikipedia in 2009 under the name ``abuse filters''.
Their clear purpose was to cope with the rising(syn) amount of vandalism as well as ``common newbie mistakes'' the encyclopedia faced~\cite{Signpost2009}.

% TODO: when and why was the extension renamed

%************************************************************************

\section{Intended Contributions}
%Epistemological interest

What do we want to know?

Core questions:

We wanted to improve our understanding of the role of filters in existing algorithmic quality-control mechanisms (bots, ORES, humans).

Which type of tasks do these filters take over in comparison to the other mechanisms? How these tasks evolve over time (are they changes in the type, number, etc.)?

Since filters are classical rule-based systems, what are suitable areas of application for such rule-based system in contrast to the other ML-based approaches.


\begin{itemize}
    \item Description of how filters integrate into the algorithmic quality control mechanism in Wikipedia
    \item Do filters work the desired way/help for a smoother Wikipedia service or is it a lot of work to maintain them and the usefulness is questionable?
    \item What can we filter with a REGEX? And what not? Are regexes the suitable technology for the means the community is trying to achieve?
    \item Filter are classical rule-based systems. What are suitable areas of application for such rule-based system in contrast to ML systems?
\end{itemize}


What can we study?

\begin{itemize}
    \item Discussions on filter patterns? On filter repercussions?
    \item Whether filters work the desired way/help for a smoother Wikipedia service or is it a lot of work to maintain them and the usefullness is questionable?
    \item What can we filter with a REGEX? And what not? Are regexes the suitable technology for the means the community is trying to achieve?
    \item add also "af\_enabled" column to filter list; could be that the high hit count was made by false positives, which will have led to disabling the filter (TODO: that's a very interesting question actually; how do we know the high number of hits were actually leggit problems the filter wanted to catch and no false positives?)
\end{itemize}


\begin{comment}
* Think about: what's the computer science take on the field? How can we design a "better"/more efficient/more user friendly system? A system that reflects particular values (vgl Code 2.0, Chapter 3, p.34)?
  * go over notes in the filter classification and think about interesting controversies, things that attract the attention
  * what are useful categories
  * GT is good for tackling controversial questions: e.g. are filters with disallow action a too severe interference with the editing process that has way too much negative consequences? (e.g. driving away new comers?)

* What can we study?
  * Question: Is it worth it to use a filter which has many side effects?
  * Do filters work the desired way/help for a smoother Wikipedia service or is it a lot of work to maintain them and the usefulness is questionable?
  * Precision/Recall: False Positives? were filters shut down, bc they matched more False positives than they had real value?
  * How has the notion of vandalism changed over time (not particularly computer sciency..)?
  * What are disallow/block filters doing? Is it too great an intervention that drives new editors away? (Can we answer that?)
  * Is it worth it to use a filter which has many side effects?
  * What are discussions on filter patterns? On filter repercussions?
  * What can we filter with a REGEX? And what not? Are regexes the suitable technology for the means the community is trying to achieve?
  * GT is good for tackling controversial questions: e.g. are filters with disallow action a too severe interference with the editing process that has way too much negative consequences? (e.g. driving away new comers?)

## TODO

* How can we improve it from a computer scientist's/engineer's perspective?
* What task do the edit filters try to solve? Why does this task exist?/Why is it important?
* Why are there mechanisms triggered befor an edit gets published (such as edit filters), and such triggered afterwards (such as bots)? Is there a qualitative difference?
* I want to help people to do their work better using a technical system (e.g. the edit filters). How can I do this?
* The edit filter system can be embedded in the vandalism prevention frame. Are there other contexts/frames for which it is relevant?
\end{comment}
Im folgenden werden Ihnen Hinweise zur Strukturierung und zum Inhalt des ersten Kapitels gegeben.

%************************************************************

\section{Subject and Context}
\begin{itemize}
	\item Wo setze ich an? (Problemstellung / Ausgangslage)
	\item Identifikation der signifikanten Problemen im betrachteten Forschungsbereich
	\item Ein kurzer Überblick über den aktuellen Forschungsstand in dem Bereich inklusive vorhandener Lösungen (ausführlicher dann in den Folgeabschnitten)
\end{itemize}

\section{Aims of this work}
\begin{itemize}
	\item Was sind die mit dieser Arbeit verfolgten Ziele? Welches Problem soll gelöst werden?
	\item Eine Beschreibung der ersten Ideen, der vorgeschlagene Ansatz und die aktuell erreichten Resultate
	\item Eine Beschreibung, welchen Beitrag die Arbeit leistet, um das vorgestellte Problem zu lösen
	\item Eine Diskussion, wie die vorgeschlagene Lösung sich von bestehenden unterscheidet, was ist neu oder besser?
\end{itemize}

\section{Methods}
\begin{itemize}
	\item Wie will ich meine Ziele erreichen? (Methodische Überlegungen)
	\item Darstellung zum Forschungsdesign.
	\item Insbesondere bei Master: Wie kann die Zielerreichung ``gemessen'' werden?
\end{itemize}

\section{Structure}
\begin{itemize}
	\item Welche Schritte werden durchlaufen, um die Ziele zu erreichen?
	\item An dieser Stelle ist beispielsweise eine Grafik hilfreich, um den Aufbau der Arbeit und welche Ergebnisse/Erkenntnisse wo genutzt werden, zu visualisieren.
	\item Ebenfalls sollten noch Anmerkungen zur Gestaltung der Arbeit gegeben werden, vor allem, da in vielen deutschen Arbeiten englische Fachbegriffe verwendet werden. Ein solcher Text könnte folgendermaßen lauten:
		\begin{itemize}
			\item ``Abschließend sind hier noch eine Anmerkungen zur Gestaltung der vorliegenden Arbeit. Für die im Folgenden verwendeten personenbezogene Ausdrücke wurde, um die Lesbarkeit der Arbeit zu erhöhen, die männliche Schreibweise gewählt. Des Weiteren werden eine Reihe von englischen Bezeichnungen verwendet, um einerseits dem interessierten Leser das Studium der häufig vorliegenden englischen Originalliteratur zu erleichtern oder andererseits bestehende Fachbegriffe nicht durch die Übersetzung zu verfälschen. Diese Begriffe sind vom herkömmlichen Text in kursiver Schrift unterschieden.''
		\end{itemize}
\end{itemize}

\begin{figure}[!ht]
	% Mit [!h] wird die Position der Grafik bestimmt. So bedeutet h=here und mit dem "!" (Ausrufezeichen) wird dieser Befehl verstärkt. Weitere Möglichkeiten sind : t=top und b=bottom. Zumeist wird angegeben, in welcher Reihenfolge LaTeX versuchen soll das Bild einzufügen, z.B. [!htb].
	\centering
		\includegraphics[width=0.95\textwidth]{pics/structure.pdf}
	\caption[Beispiel einer möglichen Darstellung zum Aufbau der Arbeit]{Beispiel einer möglichen Darstellung zum Aufbau der Arbeit (vgl. Beschreibung Abschnitt  \ref{chap:chapters}).}
	% Mit Hilfe von caption wird die Bildunterschrift erzeugt. Der Text in geschweiften Klammern erscheint im Text, während der Text in eckigen Klammern sich dann empfiehlt, wenn die Beschreibung besonders lang ist, denn diese wird dann im Bildverzeichnis verwendet. Diese Kurzbeschreibung kann auch weggelassen werden.
	\label{fig:structurethesis}
\end{figure}
