% ---------------------------------------------------
% ----- Introduction of the template
% ----- for Bachelor-, Master thesis and class papers
% ---------------------------------------------------
%  Created by C. Müller-Birn on 2012-08-17, CC-BY-SA 3.0.
%  Last upadte: C. Müller-Birn 2015-11-27
%  Freie Universität Berlin, Institute of Computer Science, Human Centered Computing.
%
\chapter{Introduction}
\label{chap:introduction}

There's an inscription in Code 2.0:
"Code 2.0
TO WIKIPEDIA, THE ONE SURPRISE THAT TEACHES MORE THAN EVERYTHING HERE." (p.v)~\cite{Lessig2006}

Comment: Wikipedia is important because it teaches us stuff. We can look at it and see things and maybe infer how other, less open systems work.

%\section{Genesis}
\begin{comment}
Don't make it a separate subsection, but use it to introduce the topic with a story, the way Geiger does.
If the genesis doesn't make sense here, move it to Edit filters
\end{comment}

% When and why were Wikipedia edit filters introduced?

Edit filters were first introduced on the English Wikipedia in 2009 under the name ``abuse filters''.
Their clear purpose was to cope with the rising(syn) amount of vandalism as well as ``common newbie mistakes'' the encyclopedia faced~\cite{Signpost2009}.

% TODO: when and why was the extension renamed

%************************************************************************
\section{Subject and Context}
%TODO should this be its own section? Or rather a part of next one?
\begin{itemize}
	\item Wo setze ich an? (Problemstellung / Ausgangslage)
	\item Identifikation der signifikanten Problemen im betrachteten Forschungsbereich
	\item Ein kurzer Überblick über den aktuellen Forschungsstand in dem Bereich inklusive vorhandener Lösungen (ausführlicher dann in den Folgeabschnitten)
\end{itemize}
%************************************************************************

The present work can be embedded in the context of (algorithmic) quality-control mechanisms on Wikipedia.
There is a whole ecosystem (syn?) of actors struggling to maintain the anyone-can-edit encyclopedia as good^^ and free of malicious, spam and ? content as possible.
We want to be able to better understand the role of edit filters in the vandal fighting network of humans, bots, semi-automated tools, and the machine learning framework ORES.
After all, edit filters were introduced to Wikipedia at a time when bots and semi-automated tools already existed and were involved in quality control: in 2009 (compare timeline, Twinkle's page is from Jan 2007, Huggle's from beginning of 2008; bot's have been around longer, but first records, at least by me so far, of vandal fighting bots come from 2006 ). %TODO: when was the other stuff introduced
Moreover, there seems to be a gap in the scientific literature on the subject.

\section{Aims of this work}
%alt title: \section{Intended Contributions}
%Epistemological interest

The aim of this work is to find out why edit filters were introduced on Wikipedia and how these fit in Wikipedia's quality control ecosystem.
More precisely, we want to unearth the tasks taken over by filters in contrast to other quality control meachanisms
and understand how different users of Wikipedia (admins/sysops, regular editors, readers) interact with these and what repercussions the filters have on them.
To this end, we study the academic contributions on Wikipedia's quality control mechanisms and give a descriptive overview of the adoption process as well as the current state of edit filters on EN Wikipedia.


\begin{comment}
Questions from Confluence
  Q1 We wanted to improve our understanding of the role of filters in existing algorithmic quality-control mechanisms (bots, ORES, humans).
  Q2 Which type of tasks do these filters take over in comparison to the other mechanisms? How these tasks evolve over time (are they changes in the type, number, etc.)?
  Q3 Since filters are classical rule-based systems, what are suitable areas of application for such rule-based system in contrast to the other ML-based approaches.
\begin{itemize}
	\item Was sind die mit dieser Arbeit verfolgten Ziele? Welches Problem soll gelöst werden?
	\item Eine Beschreibung der ersten Ideen, der vorgeschlagene Ansatz und die aktuell erreichten Resultate
	\item Eine Beschreibung, welchen Beitrag die Arbeit leistet, um das vorgestellte Problem zu lösen
	\item Eine Diskussion, wie die vorgeschlagene Lösung sich von bestehenden unterscheidet, was ist neu oder besser?
\end{itemize}

* Think about: what's the computer science take on the field? How can we design a "better"/more efficient/more user friendly system? A system that reflects particular values (vgl Code 2.0, Chapter 3, p.34)?
  * GT is good for tackling controversial questions: e.g. are filters with disallow action a too severe interference with the editing process that has way too much negative consequences? (e.g. driving away new comers?)
\end{comment}

%************************************************************

\section{Methods}
\begin{itemize}
	\item Wie will ich meine Ziele erreichen? (Methodische Überlegungen)
	\item Darstellung zum Forschungsdesign.
	\item Insbesondere bei Master: Wie kann die Zielerreichung ``gemessen'' werden?
\end{itemize}


\section{Structure}

The remaining part of this thesis is organised in the following manner:
Chapter~\ref{chap:background} situates the topic in the academic discourse and examines some key notions relevant for the subsequent analysis.
In chapter~\ref{chap:methods}, I discuss scientific methods that helped me to accomplish the analysis (syn!).
Next, I describe the edit filter mechanism in general: how and why it was conceived, how it looks like/works from from a technical and a social governance perspective.
A detailed analysis (syn!) of the current state of all implemented edit filters on English Wikipedia follows (syn) is presented in chapter~\ref{chap:overview-en-wiki}.
We discuss the findings and the limitations of the inquiry in chapter~\ref{chap:discussion}.
Finally, the analysis (syn!) is wrapped up in the conclusion where also directions for possible future investigations are given.

