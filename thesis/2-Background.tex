\chapter{Background}
\label{chap:background}

\section{Vandalism on Wikipedia}

According to Wikipedia's newspaper, the Signpost, edit filters were initially introduced as a vandalism prevention mechanism (one of several)~\cite{Signpost2009}.
The aim of this section is to provide a better understanding of vandalism on Wikipedia. (What is vandalism, and what not; who engages in vandalism; who is striving to prevent it and with what means)

%What is vandalism

According to EN Wikipedia's policy~\cite{Wikipedia:Vandalism}, vandalism means ``intentionally making abusive edits to Wikipedia'' or, more specifically ``editing (or other behavior) deliberately intended to obstruct or defeat the project's purpose, which is to create a free encyclopedia''.
Vandalism includes ``malicious removal of encyclopedic content, or the changing of such content beyond all recognition, without any regard to our core content policies of neutral point of view (which does not mean no point of view), verifiability and no original research''
as well as ``adding irrelevant obscenities or crude humor to a page, illegitimately blanking pages, and inserting obvious nonsense into a page''
and ``[a]busive creation or usage of user accounts and IP addresses''.

Wikipedians have elaborated a whole vandalism typology~\cite{Wikipedia:Vandalism}, illustrated by figure~\ref{fig:vandalism-typology}.
\begin{comment}
Types of vandalism \url{https://en.wikipedia.org/wiki/Wikipedia:Vandalism#Types_of_vandalism}:
  (Abuse of tags; Account creation, malicious; Avoidant vandalism; Blanking, illegitimate; Copyrighted material, repeated uploading of; Edit summary vandalism; Format vandalism; Gaming the system; Hidden vandalism; Hoaxing vandalism; Image vandalism; Link vandalism; Page creation, illegitimate; Page lengthening; Page-move vandalism; Silly vandalism; Sneaky vandalism; Spam external linking; Stockbroking vandalism; talk page vandalism; Template vandalism; User and user talk page vandalism; Vandalbots;)
\end{comment}

%What is not vandalism

There are different types of edits viewed as disruptive by the Wikipedia community.
Edit warring and pushing a single point of view and disregarding community feedback are examples here of. %TODO what are other examples?
Nevertheless, the guidelines caution that ``[d]isruptive editing is not vandalism, though vandalism is disruptive''~\cite{Wikipedia:DisruptiveEditing}.
And that different procedures should be adopted by editors in both cases.

The vandalism policy also cautions about using the ``vandalism'' label since it tends to drive contributors away and prevent constructive discussions~\cite{Wikipedia:Vandalism}.
%TODO vgl good faith memo

\begin{comment}
\url{https://en.wikipedia.org/wiki/Wikipedia:Vandalism}
"Careful consideration may be required to differentiate between edits that are beneficial, edits that are detrimental but well-intentioned, and edits that are vandalism."
%TODO vgl with memo-good-faith

\url{https://en.wikipedia.org/wiki/Wikipedia:Disruptive_editing}

"Disruptive editing is not always intentional. Editors may be accidentally disruptive because they don't understand how to correctly edit, or because they lack the social skills or competence necessary to work collaboratively "
Okay what are disruptive edits that are not vandalism? (apart from edit wars)

"Engages in "disruptive cite-tagging"; adds unjustified {{citation needed}} tags to an article when the content tagged is already sourced, uses such tags to suggest that properly sourced article content is questionable."
\end{comment}

%Who engages in vandalism (and why?)

The policy signals clearly that editors repeatedly engaging in vandalism are subject to banning.
Furthermore, it is explained that although warnings for vandalism are issued in general, these are not a prerequisite for banning~\cite{Wikipedia:Vandalism}.
%TODO: still not explained who and why

%Who is striving to prevent vandalism? How do they go about it?

Since Wikipedia is a ``do-it-yourself'' project, every editor who notices vandalism is called upon to help fixing it.
There is a formal process for reporting users who engage in vandalism %TODO look up Administrator intervention against vandalism
and requesting page protection for frequently vandalised pages. %TODO quote
And there are also users who specifically dedicate substantial amount of their Wikipedia contributions to fighting vandalism.

These dedicated vandal fighters mostly do so with the aid of some (semi or fully) automated tools which significally speeds up the process (see below).

\section{Quality-control mechanisms on Wikipedia}
%Context
Context of work: algorithmic quality-control mechanisms (bots, ORES, humans) -> filter?

%TODO Literature review!
Distinction filters/Bots: what tasks are handled by bots and what by filters (and why)? What difference does it make for admins? For users whose edits are being targeted?

socio-technical assemblages (see Geiger)

* Huggle, Twinkle, AWB, Bots exist nearly since the very beginning (2002?), why did the community introduce filters in 2009?

\subsection{Humans}

Some of the quality control work is done ``manually'' by humand editors.
That means, they engage in the standard encyclopedia editing mechanism (click the ``edit'' button on an article, enter changes in the editor which opens, write an edit summary for the edit, click ``save'') rather than using further automated tools to aid them.
According to research focusing on vandalism fighting, the amount/share/proportion of editors who engage in counter-vandalism measures that way shrinks in favour of semi or fully automated tools. %TODO quotes!
* what part of the quality control work do humans take over? (in contrast to the algorithmic mechanisms)

\url{https://en.wikipedia.org/wiki/Wikipedia:Recent_changes_patrol}

\subsection{Semi-automated tools}

Huggle, Twinkle, STiki~\cite{WestKanLee2010}
\url{http://en.wikipedia.org/wiki/Wikipedia:STiki}

\url{https://en.wikipedia.org/wiki/Wikipedia:AutoWikiBrowser}

\url{https://en.wikipedia.org/wiki/User:Lupin/Anti-vandal_tool}
"Please be aware that the original author of AVT (Lupin) is no longer active on Wikipedia. The script is very old and might stop working at any time."
"By using the RC feed to check a wiki-page's differences against a list of common vandal terms, this tool will detect many of the commonly known acts of online vandalism. "

\cite{GeiHal2013}
"Huggle, the most widely-used, fully assisted, counter-
vandalism tool, were made within 1 minute of the
offending edit. It is interesting that reverts with STiki, a
newer and more sophisticated queue-based vandal fighting
tool, are more often made to somewhat older edits, with a
time-to-revert distribution that is closer to unassisted edits.
This suggests that Huggle and STiki are targeting different
kinds of edits"

They also suggest that Twinkle (on one side) and Huggle and STiki are not in the same class of semi-automated vandal fighting tools, with Twinkle beeing more "manual" than the other 2.

VandalProof~\cite{HalRied2012}

"Huggle, one of the most popular
antivanda lism editing tools on
Wikipedia, is written in C#.NET
and any user can download and
install it. Huggle lets editors roll back
changes with a single mouse click,
but because the tool is so powerful,
rollback permission is restricted to
administrators and a few thousand
other Wikipedia users."
"Huggle makes it easy to review
a series of recent revisions by
filtering them according to the
user’s preferences."~\cite{HalRied2012}

huggle also sends out warnings to the offending editor on revert~\cite{HalRied2012}

\subsection{Bots}

ClueBot NG
"ClueBot_NG uses state-of-the-art machine learning techniques to review all contributions to
articles and to revert vandalism,"~\cite{HalRied2012}
XLinkBot
"XLinkBot reverts contributions that create links to
blacklisted domains as a way of quickly and permanently dealing with spammers."~\cite{HalRied2012}
HBC AIV Helperbots and MartinBot
"AIV Helperbot turns a simple page into a dynamic
priority-based discussion queue to support administrators in their work of identifying and
blocking vandals"~\cite{HalRied2012}

AntiVandalBot~\cite{HalRied2012}

Bots not patrolling constantly but instead doing batch cleanup works~\cite{GeiHal2013}:
AWB, DumbBOT, EmausBot
(also from figures: VolkovBot, WikitanvirBot, Xqbot)

\subsection{ORES}

%\section{Harassment and bullying}

\section{Algorithmic Governance}

maybe move it to edit filters chapter

\begin{itemize}
    \item Hier sollte enthalten sein, welche Anwendungen in diesem Bereich bereits existieren und warum bei diesen ein Defizit besteht.
    \item Falls genutzt, sollten hier die entsprechenden Algorithmen erläutert werden.
    \item Es sollten die Ziele der Anwendungsentwicklung, d.h. die Anforderungen herausgearbeitet werden. Dabei sollte die bestehende Literatur geeignet integriert werden.
\end{itemize}
